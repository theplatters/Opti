\documentclass{article}
\usepackage{graphicx} % Required for inserting images
\usepackage{matlab-prettifier}
\usepackage{amsmath}
\usepackage{amssymb}
\usepackage{textcomp,xcolor}
\definecolor{MatlabCellColour}{RGB}{252,251,220}
\usepackage{listings}
\definecolor{mygreen}{rgb}{0,0.6,0}
\definecolor{mygray}{rgb}{0.5,0.5,0.5}
\definecolor{mymauve}{rgb}{0.58,0,0.82}

\usepackage{geometry,lipsum}% http://ctan.org/pkg/{geometry,lipsum}

% default
\geometry{margin=3in}% 1in margin
\geometry{margin=2.5cm}% 1cm margin


\lstset{ 
	backgroundcolor=\color{white},   % choose the background color; you must add \usepackage{color} or \usepackage{xcolor}; should come as last argument
	basicstyle=\footnotesize,        % the size of the fonts that are used for the code
	breakatwhitespace=false,         % sets if automatic breaks should only happen at whitespace
	breaklines=true,                 % sets automatic line breaking
	captionpos=b,                    % sets the caption-position to bottom
	commentstyle=\color{mygreen},    % comment style
	deletekeywords={...},            % if you want to delete keywords from the given language
	escapeinside={\%*}{*)},          % if you want to add LaTeX within your code
	extendedchars=true,              % lets you use non-ASCII characters; for 8-bits encodings only, does not work with UTF-8
	frame=single,	                   % adds a frame around the code
	keepspaces=true,                 % keeps spaces in text, useful for keeping indentation of code (possibly needs columns=flexible)
	keywordstyle=\color{blue},       % keyword style
	language=Matlab,                 % the language of the code
	morekeywords={*,...},            % if you want to add more keywords to the set
	numbers=left,                    % where to put the line-numbers; possible values are (none, left, right)
	numbersep=5pt,                   % how far the line-numbers are from the code
	numberstyle=\tiny\color{mygray}, % the style that is used for the line-numbers
	rulecolor=\color{black},         % if not set, the frame-color may be changed on line-breaks within not-black text (e.g. comments (green here))
	showspaces=false,                % show spaces everywhere adding particular underscores; it overrides 'showstringspaces'
	showstringspaces=false,          % underline spaces within strings only
	showtabs=false,                  % show tabs within strings adding particular underscores 
	stepnumber=1,                    % the step between two line-numbers. If it's 1, each line will be numbered
	stringstyle=\color{mymauve},     % string literal style
	tabsize=4,	                   % sets default tabsize to 2 spaces
	title=\lstname                   % show the filename of files included with \lstinputlisting; also try caption instead of title
}

\title{Optimization Practical Exercise II\\ Sommersemester 2023}
\author{Donnermair Maximilian k12005908@students.jku.at\\ Fromherz Jakob k12011689@students.jku.at\\Haslhofer Eva-Maria  k12007773@students.jku.at \\ Scharnreitner Franz k12011695@students.jku.at\\ Weiss Hannah Maria k12021111@students.jku.at } %Bitte Matrikelnummer in Email einfügen
\date{30 June 2023}

%Bitte im Folder Code die Programmfiles am Schluss aktualisieren
\begin{document}
	\maketitle
	
	\newpage
	
	\section{Exercise 3}
	\subsection{Sourcecode}
	\lstinputlisting{src/PractEx2linprog.m}
	\subsection{Output}
	\lstinputlisting{src/PractEx2linprogOut.txt}
	\subsection{Interpretation}
	Note that \texttt{MATLAB} uses a different definition for the inequality constraints so that we needed to multiply with the factor $-1$.\\
	For the interior-point algorithm Matlab returns the vector
	$$ \overline{x} = \begin{pmatrix}
		0.5589\\
		0.6473\\
		0.6178\\
		0
	\end{pmatrix}$$
	which is different from the solution of exercise 31 given by
	$$\overline{x} = \begin{pmatrix}
	\frac{1}{2}\\
	\frac{1}{2}\\
	\frac{1}{2}\\
	0
	\end{pmatrix}.$$
	Yet, both vectors give the same result in function evaluation. In contrast to the interior-point algorithm, the dual-simplex one returns the more deviating solution 
	$$\overline{x} = \begin{pmatrix}
	\frac{1}{3}\\
	\frac{1}{12}\\
	\frac{1}{6}\\
	0
	\end{pmatrix}.$$
	Moreover, it takes only two instead of three iterations to terminate as this is the case for the interior-point algorithm.\\
	Even though the solutions of both algorithms differ the Lagrange multiplicators do not, i.e. in both cases we obtain
	$$ \overline{\lambda} = \begin{pmatrix}
	-2\\
	1\\
	0
	\end{pmatrix}.$$
	This result also matches the Lagrange multiplicators calculated in exercise 31.
	
	\section{Exercise 4}
	\subsection{Sourcecode}
	\lstinputlisting{src/PractEx2mincon.m}
	\subsection{Output}
	\lstinputlisting{src/PractEx2minconOut.txt}
	\subsection{Interpretation}
	MATLAB returns a solution for all three n $\in \{5,10,100\}$, but for $n = 100$, an increase of the maximum number of function evaluations was necessary. This is because it did not return a solution for the default number of evaluations.
	
\section{Exercise 5}
Implementation of an interior point algorithm (compare algorithm 8.2).



\end{document}