\documentclass{article}
\usepackage{graphicx} % Required for inserting images
\usepackage{matlab-prettifier}
\usepackage{amsmath}
\usepackage{amssymb}
\usepackage{textcomp,xcolor}
\definecolor{MatlabCellColour}{RGB}{252,251,220}
\usepackage{listings}
\definecolor{mygreen}{rgb}{0,0.6,0}
\definecolor{mygray}{rgb}{0.5,0.5,0.5}
\definecolor{mymauve}{rgb}{0.58,0,0.82}

\usepackage{geometry,lipsum}% http://ctan.org/pkg/{geometry,lipsum}

% default
\geometry{margin=3in}% 1in margin
\geometry{margin=2.5cm}% 1cm margin


\lstset{ 
	backgroundcolor=\color{white},   % choose the background color; you must add \usepackage{color} or \usepackage{xcolor}; should come as last argument
	basicstyle=\footnotesize,        % the size of the fonts that are used for the code
	breakatwhitespace=false,         % sets if automatic breaks should only happen at whitespace
	breaklines=true,                 % sets automatic line breaking
	captionpos=b,                    % sets the caption-position to bottom
	commentstyle=\color{mygreen},    % comment style
	deletekeywords={...},            % if you want to delete keywords from the given language
	escapeinside={\%*}{*)},          % if you want to add LaTeX within your code
	extendedchars=true,              % lets you use non-ASCII characters; for 8-bits encodings only, does not work with UTF-8
	frame=single,	                   % adds a frame around the code
	keepspaces=true,                 % keeps spaces in text, useful for keeping indentation of code (possibly needs columns=flexible)
	keywordstyle=\color{blue},       % keyword style
	language=Matlab,                 % the language of the code
	morekeywords={*,...},            % if you want to add more keywords to the set
	numbers=left,                    % where to put the line-numbers; possible values are (none, left, right)
	numbersep=5pt,                   % how far the line-numbers are from the code
	numberstyle=\tiny\color{mygray}, % the style that is used for the line-numbers
	rulecolor=\color{black},         % if not set, the frame-color may be changed on line-breaks within not-black text (e.g. comments (green here))
	showspaces=false,                % show spaces everywhere adding particular underscores; it overrides 'showstringspaces'
	showstringspaces=false,          % underline spaces within strings only
	showtabs=false,                  % show tabs within strings adding particular underscores 
	stepnumber=1,                    % the step between two line-numbers. If it's 1, each line will be numbered
	stringstyle=\color{mymauve},     % string literal style
	tabsize=4,	                   % sets default tabsize to 2 spaces
	title=\lstname                   % show the filename of files included with \lstinputlisting; also try caption instead of title
}

\title{Optimization Practical Exercise II\\ Sommersemester 2023}
\author{Donnermair Maximilian k12005908@students.jku.at\\ Fromherz Jakob k12011689@students.jku.at\\Haslhofer Eva-Maria  k12007773@students.jku.at \\ Scharnreitner Franz k12011695@students.jku.at\\ Weiss Hannah Maria k12021111@students.jku.at } %Bitte Matrikelnummer in Email einfügen
\date{30 June 2023}

%Bitte im Folder Code die Programmfiles am Schluss aktualisieren
\begin{document}
	\maketitle
	
	\newpage
	
	\section{Exercise 3}
	\subsection{Sourcecode}
	\lstinputlisting{src/PractEx2linprog.m}
	\subsection{Output}
	\lstinputlisting{src/PractEx2linprogOut.txt}
	\subsection{Interpretation}
	For the interior-point algorithm Matlab returns the vector
	$$ \overline{x} = \begin{pmatrix}
		0.5589\\
		0.6473\\
		0.6178\\
		0
	\end{pmatrix},$$
	which is different from the solution of exercise 31 given by
	$$\overline{x} = \begin{pmatrix}
	\frac{1}{2}\\
	\frac{1}{2}\\
	\frac{1}{2}\\
	0
	\end{pmatrix}.$$
	Yet, both vectors give the same result in function evaluation. In contrast to the interior-point algorithm, the dual-simplex one returns the more deviating solution: 
	$$\overline{x} = \begin{pmatrix}
	\frac{1}{3}\\
	\frac{1}{12}\\
	\frac{1}{6}\\
	0
	\end{pmatrix}.$$
	Moreover, it takes only two instead of three iterations to terminate as this is the case for the interior-point algorithm.\\
	Even though the primal solutions of both algorithms differ, the Lagrange multiplicators do not, i.e. in both cases we obtain
	$$ \overline{\lambda} = \begin{pmatrix}
	2\\
	1\\
	0
	\end{pmatrix}.$$
    It is, however, different to the multiplicators we obtained in Exercise 31 by applying Theorem 7.10:
    $$ \overline{\lambda} = \begin{pmatrix}
	-2\\
	1\\
	0
	\end{pmatrix}.$$
	The sign difference stems from the fact that \texttt{MATLAB} states in its linprog-documentation that their sign convention for the lambda is the opposite of much of the literature.
	
    \section{Exercise 4}
	\subsection{Sourcecode}
	\lstinputlisting{src/PractEx2mincon.m}
	\subsection{Output}
	\lstinputlisting{src/PractEx2minconOut.txt}
	\subsection{Interpretation}
	MATLAB returns a solution for all three n $\in \{5,10,100\}$, but for $n = 100$, an increase of the maximum number of function evaluations was necessary. This is because it did not return a solution for the default number of evaluations.
	
\section{Exercise 5}
Implementation of an interior point algorithm (compare algorithm 8.2) to solve quadratic problems. As described in the script,  algorithm tries to solve the Problem $QP(\mu)$ with solutions $(x(\mu),\lambda(\mu))$. We approximate $\mu=0$, to get an approximation of the solution $(x(0),\lambda(0))$.
\subsection{Sourcecode}
\lstinputlisting{src/interiorpoint.m}
\lstinputlisting{src/gradLagrangian.m}
\lstinputlisting{src/corrector.m}
\lstinputlisting{src/predictor.m}
\lstinputlisting{src/isInVa.m}

\subsection{Output}

\subsubsection{Testing with Exercise 31}
\lstinputlisting{src/ex5TestEx31.m}
\lstinputlisting{src/ex5TestEx31.txt}
\subsubsection{Interpretation}
We tested our implementation of the interior-point algorithm with the Exercise 31. The problem stated in this exercise is a linear problem, therefore G=0, which is a positive semi-definite matrix. As a starting value, we used $x_0=(0,0,0,1)^T$ and chose different values for $\mu_0 \in \{0.1,0.5,1,100\}$ and $e \in \{0.1,1,100\}$, which is a deviation variable, that increases the strength of the inequality $a_i^T x < b_i ´+ \beta_i \mu_0$. In every test we converged to a solution (exitflag=0). In the case that $\mu_0=0.5$ and the deviation parameter is 1, we reached the solution $x \approx (0.5,0.5,0.5,0)^T$, which we showed in Exercise 37 to be a solution. For other $\mu_0$ and deviation parameters, we reached other solutions. For a high deviation variable and a high $\mu_0$ we get the warning, that the matrix is badly scaled. The iteration number is quite low (at most than 60 iterations are needed) and the computation time is low (needed only 0.153727 seconds for all tests with even displaying the output). With  outputViolations = max(abs(min(b - A'*x,0))) we calculated the maximal violation of the constraints. We see that these are small. If $\mu_0$ is getting smaller and still converging to the same solution, the violation of the constraints gets smaller.      
\subsubsection{Testing with other examples}
Here we test \texttt{interiorpoint} with the example function provided. different mu different dimensions. interiorpoint outputs minimizer, lagr. multipliers,exitflag, whether interiorpoint found minimizer under the specified convergence criteria.
\lstinputlisting{src/ex5Test.m} 
Below, the script \texttt{testFunction.m} creates the local objects for the function to to be minimized under the specified constraints, that are then passed on to \texttt{interiorpoint} in dependence of a dimensionality variable $N$.
\lstinputlisting{src/testFunction.m}
Below, find the output of \texttt{ex5Test.m}. 
\lstinputlisting{src/ex5Test.txt}
\end{document}